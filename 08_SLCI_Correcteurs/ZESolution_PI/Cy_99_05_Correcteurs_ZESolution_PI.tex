\documentclass[10pt,fleqn]{article} % Default font size and left-justified equations
\usepackage[%
    pdftitle={Modélisation dynamique},
    pdfauthor={Xavier Pessoles}]{hyperref}

    
\input{style/new_style}
\input{style/macros_SII}
\usepackage{multicol}
\usepackage{siunitx}
%\usepackage{picins}
\fichetrue
%\fichefalse

\proftrue
\proffalse

\tdtrue
%\tdfalse

\courstrue
\coursfalse


\def\classe{\textsf{PSI$\star$ -- MP}}
\def\xxnumpartie{Cycle 99}
\def\xxpartie{Révisions}

\def\xxnumchapitre{}%Chapitre 1 \vspace{.2cm}}
\def\xxchapitre{}%\hspace{.12cm} Approche énergétique}

\def\discipline{Sciences \\Industrielles de \\ l'Ingénieur}
\def\xxtete{Sciences Industrielles de l'Ingénieur}




\def\xxtitreexo{Bras manipulateur collaboratif ZE Solution}
\def\xxsourceexo{\hspace{.2cm} \textsl{Concours Centrale Supelec MP 2016}
}


\def\xxposongletx{2}
\def\xxposonglettext{1.45}
\def\xxposonglety{20}
%\def\xxonglet{Part. 1 -- Ch. 3}
\def\xxonglet{\textsf{Cycle 99}}

\def\xxactivite{Application}
\def\xxauteur{\textsl{Xavier Pessoles}}

\def\xxcompetences{%
\textsl{%
\textbf{Savoirs et compétences :}\\
%Les sources sont associées par un \emph{hacheur série}. La détermination des grandeurs électriques associées à ce montage permet de conclure vis à vis du cahier des charges.
%\noindent \textbf{Résoudre :} à partir des modèles retenus :
%\begin{itemize}[label=\ding{112},font=\color{ocre}] 
%\item choisir une méthode de résolution analytique, graphique, numérique;
%\item mettre en \oe{}uvre une méthode de résolution.
%\end{itemize}
%\begin{itemize}[label=\ding{112},font=\color{ocre}] 
%\item \textit{Rés -- C1.1 :} Loi entrée sortie géométrique et cinématique -- Fermeture géométrique.
%\end{itemize}
%
%\noindent \textit{Mod2 -- C4.1 :} Représentation par schéma bloc.
}}

\def\xxfigures{
\includegraphics[width=.7\linewidth]{images/fig_00}
}%figues de la page de garde


\def\xxpied{%
Cycle 99 -- Révisions\\% afin de valider leurs performances.\\
Bras manipulateur collaboratif ZE Solution %-- \xxactivite%
}

\setcounter{secnumdepth}{5}
%---------------------------------------------------------------------------

\usepackage{pgfplots}
\begin{document}
\def\pathfig{images}
%\chapterimage{png/Fond_Cin}
\input{style/new_pagegarde}
\vspace{4.5cm}
\pagestyle{fancy}
\thispagestyle{plain}

\def\columnseprulecolor{\color{ocre}}
\setlength{\columnseprule}{0.4pt} 

\def\pathfig{images}

\ifprof
%\begin{multicols}{2}
\else
\begin{multicols}{2}
\fi


\subsection*{Présentation du système}
\setcounter{subparagraph}{0}
Le bras manipulateur collaboratif ZE Solution est un bras collaboratif permettant d'assister des opérateurs à manipuler des charges allant jusqu'à \SI{200}{kg} << sans effort >>. 


\begin{center}
\includegraphics[width=\linewidth]{images/fig_01}
%\textit{Illustration du pilotage à distance du système ROBOVOLC}
\end{center}

Le schéma-blocs de la chaîne de motorisation asservie est présenté sur le document réponse.

Le calculateur de la boucle de vitesse génère la consigne de courant d’intensité $i_c(t)$ à imposer au moteur brushless en comparant la vitesse angulaire de consigne à la vitesse angulaire réelle mesurée par un capteur de vitesse placé sur l’arbre moteur.

La boucle interne de courant du moteur brushless est considérée parfaite et en conséquence est modélisée par un gain unitaire, comme indiquée dans le document réponse.

Le couple $C_m(t)$ fourni par le moteur brushless au réducteur vérifie la relation $C_m(t)=K_mi(t)$.

\textbf{Paramètres géométriques et cinématiques :}
\begin{itemize}
\item vitesse angulaire du moteur $\omega_m(t)\vect{z}$;
\item vitesse angulaire du tambour $\omega_T(t)\vect{z}$;
\item rayon du tambour et de la poulie de renvoi : $R=\SI{0,05}{m}$;
\item rapport de réduction du réducteur : $\dfrac{1}{\rho}=\dfrac{\omega_T(t)}{\omega_m(t)}$ avec $\rho=15,88$;
\item vitesse linéaire de la masse en translation : $v(t)\vect{z}=\dot{z}\vect{z}$.
\end{itemize}

\textbf{Paramètres massiques et d'inertie :}
\begin{itemize}
\item masse entraînée $M$ de centre de gravité $G$, $M\leq \SI{200}{kg}$;
\item inertie du moto-réducteur autour de son axe de rotation, rapportée sur l’axe du moteur brushless $J_0=\SI{0,00315}{kg.m^2}$.
\end{itemize}

\textbf{Le cahier des charges est le suivant : }
\begin{itemize}
\item stabilité : $M_{\varphi}=80\degres$;
\item précision : 
\begin{itemize}
\item écart statique permanent vis-à-vis d'une entrée en échelon : nul;
\item écart statique permanent vis-à-vis de la perturbation (B) constante : nul;
\end{itemize}
\item rapidité : pulsation de coupure à \SI{0}{dB} de la fonction de transfert en boucle ouverte : $\omega_{\SI{0}{dB}}=\SI{40}{rad.s^{-1}}$.
\end{itemize}
On note $\omega_{mc}(t)$ la vitesse angulaire de consigne et $C_v(p)$ la fonction de transfert du correcteur de la boucle de vitesse.

On pose $v(t)=K_r \omega_m(t)=$ avec $K_r>0$ par convention.
\subparagraph{}
\textit{Déterminer la valeur numérique de $K_r$.}%. Vérifier que l’actionneur retenu permet de respecter l’exigence id 1.4.1.

L'application du théorème de l'énergie cinétique permet de montrer que $A\dot{\omega}_m(t)=C_m(t)-B(t)$ avec $A=J_0+MK_r^2$ avec $B(t)=MgK_r u(t)$.

\subparagraph{}
\textit{Compléter le schéma-blocs du document réponses (dans le domaine de Laplace) en fonction des paramètres $A$, $\rho$, $R$ et $K_m$. Les conditions initiales sont supposées nulles.}

Pour une charge à déplacer de $M=\SI{100}{kg}$, la fonction de transfert en boucle ouverte non corrigée (c’est-à-dire en considérant $C_v(p)=1$) vaut numériquement : $\text{FTBO}_{\text{nc}}(p)=\dfrac{K_{\text{BO}}}{p}$ avec $K_{\text{BO}}=218$.

\subparagraph{}
\textit{Déterminer la marge de phase $M_{\varphi}$. Est-ce satisfaisant vis-à-vis du critère de stabilité du cahier des charges ?}

Le constructeur choisit un correcteur de type Proportionnel Intégral : $C_v(p)= K_i\left(1+ \dfrac{1}{T_i p}\right)$.

\subparagraph{}
\textit{Quelle performance du cahier des charges justifie l’utilisation de ce type de correcteur ? Tracer le diagramme de Bode asymptotique du correcteur $C_v(p)$ en précisant les points caractéristiques en fonction de $K_i$ et $T_i$.}


\subparagraph{}
\textit{Déterminer la valeur numérique de $T_i$ afin que la marge de phase du système corrigé soit exactement de 80\degres  tout en respectant l’exigence de rapidité du cahier des charges. En déduire alors la valeur de $K_i$ permettant
de satisfaire l’ensemble des performances attendues dans le cahier des charges.}


Les réponses temporelles simulée et réelle de la chaîne de motorisation asservie (avec une charge à déplacer de masse $M=\SI{100}{kg}$) à un échelon de vitesse de consigne de \SI{5 000}{tr.min^{-1}} avec les réglages déterminés
précédemment sont fournies sur la figure suivante.


\begin{center}
\includegraphics[width=\linewidth]{images/fig_03}
%\textit{Illustration du pilotage à distance du système ROBOVOLC}
\end{center}

\subparagraph{}
\textit{Comparer les résultats simulés et les résultats expérimentaux sur les critères du temps de réponse à 5\%
et de la valeur finale.}



\end{multicols}

\begin{center}
\includegraphics[width=\linewidth]{images/fig_02}
%\textit{Illustration du pilotage à distance du système ROBOVOLC}
\end{center}


\end{document}	


\subparagraph{}
\textit{}
\ifprof
\begin{corrige}
\end{corrige}
\else
\fi
