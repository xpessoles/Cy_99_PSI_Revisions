\documentclass[10pt,fleqn]{article} % Default font size and left-justified equations
\usepackage[%
    pdftitle={Modélisation dynamique},
    pdfauthor={Xavier Pessoles}]{hyperref}

    
\input{style/new_style}
\input{style/macros_SII}
\usepackage{multicol}
\usepackage{siunitx}
%\usepackage{picins}
\fichetrue
%\fichefalse

\proftrue
%\proffalse

\tdtrue
%\tdfalse

\courstrue
\coursfalse


\def\classe{\textsf{PSI$\star$ -- MP}}
\def\xxnumpartie{Cycle 99}
\def\xxpartie{Révisions}

\def\xxnumchapitre{}%Chapitre 1 \vspace{.2cm}}
\def\xxchapitre{}%\hspace{.12cm} Approche énergétique}

\def\discipline{Sciences \\Industrielles de \\ l'Ingénieur}
\def\xxtete{Sciences Industrielles de l'Ingénieur}




\def\xxtitreexo{Gouverne de profondeur d'un airbus}
\def\xxsourceexo{\hspace{.2cm} \textsl{}}%Équipe PT La Martinière Monplaisir -- X ENS PSI}



\def\xxposongletx{2}
\def\xxposonglettext{1.45}
\def\xxposonglety{20}
%\def\xxonglet{Part. 1 -- Ch. 3}
\def\xxonglet{\textsf{Cycle 99}}

\def\xxactivite{TP}
\def\xxauteur{\textsl{Équipe PT La Martinière Monplaisir}}

\def\xxcompetences{%
\textsl{%
\textbf{Savoirs et compétences :}\\
%Les sources sont associées par un \emph{hacheur série}. La détermination des grandeurs électriques associées à ce montage permet de conclure vis à vis du cahier des charges.
%\noindent \textbf{Résoudre :} à partir des modèles retenus :
%\begin{itemize}[label=\ding{112},font=\color{ocre}] 
%\item choisir une méthode de résolution analytique, graphique, numérique;
%\item mettre en \oe{}uvre une méthode de résolution.
%\end{itemize}
%\begin{itemize}[label=\ding{112},font=\color{ocre}] 
%\item \textit{Rés -- C1.1 :} Loi entrée sortie géométrique et cinématique -- Fermeture géométrique.
%\end{itemize}
%
%\noindent \textit{Mod2 -- C4.1 :} Représentation par schéma bloc.
}}

\def\xxfigures{
\includegraphics[width=.8\linewidth]{images/fig_00}
}%figues de la page de garde


\def\xxpied{%
Cycle 99 -- Révisions\\% afin de valider leurs performances.\\
 -- \xxactivite%
}

\setcounter{secnumdepth}{5}
%---------------------------------------------------------------------------

\usepackage{pgfplots}
\begin{document}
\def\pathfig{images}
%\chapterimage{png/Fond_Cin}
\input{style/new_pagegarde}
\vspace{4.5cm}
\pagestyle{fancy}
\thispagestyle{plain}

\def\columnseprulecolor{\color{ocre}}
\setlength{\columnseprule}{0.4pt} 

\def\pathfig{images}

\ifprof
%\begin{multicols}{2}
\else
\begin{multicols}{2}
\fi


\subsection*{Présentation du système}
\setcounter{subparagraph}{0}

On se propose d’étudier la réalisation de la fonction « Asservir en position la gouverne de profondeur ». On se limitera à l’asservissement en position de la servocommande d’une gouverne intérieure. 

\begin{center}
\includegraphics[width=\linewidth]{images/fig_01}

\includegraphics[width=\linewidth]{images/fig_02}
\end{center}

Nous allons seulement étudier l’asservissement en position de la tige du vérin (c’est-à-dire la première boucle d’asservissement) dont le cahier des charges est le suivant.

\begin{center}
\textbf{Asservir en position la tige du vérin }
\begin{tabular}{|p{.45\linewidth}|p{.45\linewidth}|}
\hline
 Exigence	& Niveaux \\ \hline\hline
 Marge de phase	 & $\geq 60\degres $\\ \hline
 Marge de gain	 & \SI{10}{dB} \\ \hline
 Ecart de position&	$\varepsilon_p=\SI{0}{mm}$\\ \hline
 Ecart de traînage pour une consigne $x_{2c} (t)=0,1 t $ & $\varepsilon_T<\SI{0.2}{mm}$\\ \hline
 Temps de réponse à 5\% (échelon)& $tR_{5\%}< \SI{0.045}{s}$ \\ \hline
 Dépassement (échelon)	& $D\%< 5\%$\\ \hline
\end{tabular}
\end{center}

La chaîne d’énergie qui permet de modifier l’inclinaison de gouverne de profondeur est composée d’une servovalve comme actionneur et d’un vérin comme effecteur. Comme le montre la figure suivante, il y a deux boucles d’asservissement pour asservir en position la gouverne de profondeur. On note $i$ l'intensité alimentant la servovalve, $Q$ le débit alimentant le vérin, $\beta$ l'inclinaison des gouvernes par rapport au PHR et $x_2$ la position de la tige du vérin.
 



\begin{center}
\includegraphics[width=\linewidth]{images/fig_03}
\end{center}


On modélise cet asservissement par le schéma bloc suivant.
\begin{center}
\includegraphics[width=\linewidth]{images/fig_04}
\end{center}
$C(p)$ est la fonction de transfert du correcteur, $FTBO_{nc}(p)$ est la fonction de transfert de la FTBO non corrigée : 
$$
FTBO_{nc}(p)=\dfrac{0,01}{p\left(1+\dfrac{2\cdot 0,0032}{161,7}p + \dfrac{p^2}{161,7^2} \right)}.
$$

\textbf{Les questions seront traitées de manière analytique en TD puis en utilisant Scilab en TP.}

\subsection*{Système non corrigé}

\subparagraph{}
\textit{Déterminer les niveaux des 6 critères du cahier des charges. Conclure.}
\ifprof
\begin{corrige}
La boucle ouverte est de classe 1. On a donc : 
\begin{itemize}
\item $\varepsilon_S=\SI{0}{mm}$ >> CDC OK;
\item $\varepsilon_T=\dfrac{0,2}{0,01}=\SI{20}{mm}$ >> CDC pas OK.
\end{itemize}
On trace le diagramme de Bode :
\begin{itemize}
\item Marge de phase : 90\degres >> CDC OK;
\item Marge de gain : \SI{40,29}{dB} >> CDC OK.
\end{itemize}

On peut aussi déterminer les marges par le calcul :
\begin{itemize}
\item module de la fonction de transfert : 
$H_{\text{dB}}(\omega)=20\log K - 20 \log \omega - 20 \log \left( \sqrt{ \left(1-\dfrac{\omega^2}{\omega_0^2} \right)^2+\left(\dfrac{2\xi\omega}{\omega_0} \right)^2}\right)$;
\item argument de la fonction de transfert : 
$\varphi(\omega)=\arg\left(0,01\right)-\arg\left(j\omega\right)-\arg\left(0,01\right)-\arg \left( \left(1-\dfrac{\omega^2}{\omega_0^2} \right)+\left(\dfrac{2\xi\omega}{\omega_0} \right)j\right)$.
\end{itemize}


On commence par résoudre $H_{\text{dB}}(\omega_{\SI{0}{dB}})=0 \Rightarrow \omega_{\SI{0}{dB}}=0,01$ et $\varphi(0,01)=-\SI{90}{\degres}$. La marge de phase est donc de $M\varphi=-\SI{90}{\degres}-\left(-\SI{180}{\degres}\right)=90\degres$.

On résout maintenant $\varphi(\omega_{\SI{180}{\degres}})=-180\degres$ et $\omega_{\SI{180}{\degres}}=\omega_0$. On calcule ensuite $H_{\text{dB}}(\omega_{\SI{180}{\degres}})=-\SI{40,29}{dB}$ soit une marge de gain de \SI{40,29}{dB}.


Le dépassement et le temps de réponse ne peuvent pas être déterminés analytiquement. 


\end{corrige}
\else
\fi

\begin{obj}
L’objectif du TP (TD) est de trouver les caractéristiques d’un correcteur qui permet de valider le cahier des charges.
\end{obj}

\subsection*{Action proportionnelle}


On choisit d’utiliser un correcteur proportionnel dont la fonction de transfert est $C(p)=K_p$.
\subparagraph{}
\textit{Trouver la plus grande valeur de $K_p$ qui permet de vérifier les marges de stabilité.}
\ifprof
\begin{corrige}
\textbf{Réglage de la marge de gain}
On a vu que pour le système non corrigé, la marge de gain était de \SI{40,29}{dB}. Pour être à la limite de la stabilité, on pourrait augmenter le gain de $20\log K_p = \SI{30,29}{dB} \Rightarrow K_p = {32,7} $. Il faut donc $K_p\leq 32,7$.

Dans ces conditions, la marge de phase vaut : 90\degres.

\textbf{Réglage de la marge de phase}
On cherche $\omega$ tel que $\varphi\left( \omega\right)=-120\degres$. On a alors $\omega=\SI{160,806}{rad.s^{-1}}$. On cherche donc $K_p $ tel que $H_{\text{dB}}(160,806)=\SI{0}{dB}$. Pour le système non corrigé, on a $H_{\text{dB}}(160,806)=-\SI{46,22}{dB}$. On a donc $K_p = 205$.

... Quasiment impossible à régler en raison de la raideur de la pente sur la phase...

\end{corrige}
\else
\fi

\subparagraph{}
\textit{Expliquer pourquoi l’écart de position (ou écart statique) ne dépend pas de la valeur de $K_p$.}
\ifprof
\begin{corrige}
BO de classe 1. 
\end{corrige}
\else
\fi

\subparagraph{}
\textit{Trouver la plus grande valeur de $K_p$ qui permet de vérifier l'écart de trainage. Conclure.}
\ifprof
\begin{corrige}
On cherche $K_p$ tel que $\dfrac{0,1}{0,01K}=0,2 \Leftrightarrow K=\dfrac{0,1}{0,01\cdot 0,2 }= 100$. Il faut donc $K_p\geq50$.
\end{corrige}
\else
\fi

\subparagraph{}
\textit{Faire un bilan, dans un tableau, de l’influence de $K_p$ (pour $K_p>1$) sur les 3 performances : stabilité, précision et rapidité. Quelles sont les performances qui vont ensemble et celles qui sont antagonistes ?}
\ifprof
\begin{corrige}
\end{corrige}
\else
\fi

\subparagraph{}
\textit{Un correcteur proportionnel suffit-il à vérifier le cahier des charges ?}
\ifprof
\begin{corrige}
\end{corrige}
\else
\fi

\subsection*{Action intégrale}
On choisit d’utiliser un correcteur intégral dont la fonction de transfert est  $C(p)=\dfrac{K_I}{p}$.
\subparagraph{}
\textit{Faire un bilan, dans un tableau, de l’influence de la présence d’un correcteur intégral sur les 2 performances : stabilité et précision.}
\ifprof
\begin{corrige}
On a maintenant une BO de classe 2 et donc l'erreur de trainage sera nulle. Cependant, l'intégrateur ajoute au système un déphasage de 90\degres ce qui se traduit par une phase qui sera toujours strictement inférieure à 180\degres. La marge de phase sera donc négative et le système sera instable. 
\end{corrige}
\else
\fi

\subsection*{Action dérivée}
On choisit d’utiliser un correcteur dérivé dont la fonction de transfert est  $C(p)=K_Dp$.
\subparagraph{}
\textit{Faire un bilan, dans un tableau, de l’influence de la présence d’un correcteur dérivé sur les 2 performances : stabilité et précision}
\ifprof
\begin{corrige}
L'apport de phase va stabiliser le système. Cependant, le système sera alors de classe nulle et il sera impossible d'avoir un système précis << statiquement >>.
\end{corrige}
\else
\fi.

\subsection*{Correcteur à avance de phase}
On choisit d’utiliser un correcteur à avance de phase dont la fonction de transfert est 
$C(p)=\dfrac{K (1+aTp)}{1+Tp}$avec $a>1$.
\subparagraph{}
\textit{Déterminer les paramètres du correcteur à avance de phase pour vérifier le critère de stabilité.}
\ifprof
\begin{corrige}
\end{corrige}
\else
\fi

\subsection*{Correcteur à retard de phase}
On choisit d’utiliser un correcteur à retard de phase dont la fonction de transfert est $C(p)=K \dfrac{1+Tp}{1+bTp}$ avec $b>1$.
\subparagraph{}
\textit{Déterminer les paramètres du correcteur à retard de phase pour vérifier le critère de précision sans impacter la stabilité du système.}
\ifprof
\begin{corrige}
\end{corrige}
\else
\fi


\subsection*{Correcteur PID}
Afin de profiter des avantages des trois actions précédentes, on utilise un correcteur Proportionnel-Intégral-Dérivé :
$C(p)=K_p+K_D p+\dfrac{K_I}{p}=K\dfrac{1+\dfrac{2\xi}{\omega_0}  p+\dfrac{p^2}{\omega_0^2}}{p}$.
  
\subparagraph{}
\textit{Que dire sur les critères de précision du cahier des charges avec ce correcteur.}
\ifprof
\begin{corrige}
\end{corrige}
\else
\fi

\subparagraph{}
\textit{En faisant varier $K$, $\xi$ et $\omega_0$, observer leur influence sur la rapidité (bande passante de la FTBF).}
\ifprof
\begin{corrige}
\end{corrige}
\else
\fi

Afin de vérifier le critère sur la marge de gain on va trouver des paramètres pour que la phase de la FTBO soit toujours supérieure à -180\degres..
\subparagraph{}
\textit{Quel est l’influence de $K$, $\xi$ et $\omega_0$, sur la phase de la FTBO ? }
\ifprof
\begin{corrige}
\end{corrige}
\else
\fi

\subparagraph{}
\textit{Trouver un couple de valeur de $K$ et $\omega_0$ permettant de vérifier le cahier des charges : proposez une méthodologie.}
\ifprof
\begin{corrige}
\end{corrige}
\else
\fi

%Repartir du système sans correction


\subsection*{Correcteur « compensateur »}
On choisit un correcteur, réalisable numériquement, de fonction de transfert :	$C(p)=K_c\dfrac{ N(p)}{D(p)}=K_c\dfrac{1+\dfrac{2\xi}{\omega_0} p+\dfrac{p^2}{\omega_0^2}}{1+\dfrac{2\xi_c}{\omega_c}  p+\dfrac{p^2}{\omega_c^2 }}$.
Caractéristiques du correcteur :	
\begin{itemize}
\item le gain $K_c$ du correcteur est choisi égal à 50 ;
\item le facteur d’amortissement $\xi_c$ est choisi égal à 0,7 ;
\item le numérateur $N(p)$ de $C(p)$ est choisi égal au terme du second ordre du dénominateur de la fonction $H(p)$.
\end{itemize}
\subparagraph{}
\textit{Justifier les choix de la valeur du gain de boucle $K_c$ et celle du facteur d’amortissement $\xi_c$.}
\ifprof
\begin{corrige}
\end{corrige}
\else
\fi

\subparagraph{}
\textit{Donner la nouvelle expression de la FTBO. Expliquer le nom de ce correcteur.}
\ifprof
\begin{corrige}
\end{corrige}
\else
\fi

\subparagraph{}
\textit{Que vaut la phase de la FTBO pour $\omega_c$ ? Pour quelles valeurs de $\omega_c$  le système est-il instable ?}
\ifprof
\begin{corrige}
\end{corrige}
\else
\fi

\subparagraph{}
\textit{Donner la valeur de $\omega_c$  qui permet de vérifier la marge de phase de 60\degres.}
\ifprof
\begin{corrige}
\end{corrige}
\else
\fi



\ifprof
%\begin{multicols}{2}
\else
\end{multicols}
\fi
\end{document}	


\subparagraph{}
\textit{}
\ifprof
\begin{corrige}
\end{corrige}
\else
\fi
