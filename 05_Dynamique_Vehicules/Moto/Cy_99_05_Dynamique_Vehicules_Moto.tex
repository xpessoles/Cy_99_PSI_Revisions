\documentclass[10pt,fleqn]{article} % Default font size and left-justified equations
\usepackage[%
    pdftitle={Modélisation dynamique},
    pdfauthor={Xavier Pessoles}]{hyperref}

    
\input{style/new_style}
\input{style/macros_SII}
\usepackage{multicol}
\usepackage{siunitx}
%\usepackage{picins}
\fichetrue
%\fichefalse

\proftrue
\proffalse

\tdtrue
%\tdfalse

\courstrue
\coursfalse


\def\classe{\textsf{PSI$\star$ -- MP}}
\def\xxnumpartie{Cycle 99}
\def\xxpartie{Révisions}

\def\xxnumchapitre{}%Chapitre 1 \vspace{.2cm}}
\def\xxchapitre{}%\hspace{.12cm} Approche énergétique}

\def\discipline{Sciences \\Industrielles de \\ l'Ingénieur}
\def\xxtete{Sciences Industrielles de l'Ingénieur}




\def\xxtitreexo{Analyse du comportement dynamique global d'une moto}%Motorisation du moteur Haibike}
\def\xxsourceexo{\hspace{.2cm} \textsl{Équipe PT La Martinière Monplaisir}
}


\def\xxposongletx{2}
\def\xxposonglettext{1.45}
\def\xxposonglety{20}
%\def\xxonglet{Part. 1 -- Ch. 3}
\def\xxonglet{Cycle 02}

\def\xxactivite{Application}
\def\xxauteur{\textsl{Équipe PT La Martinière Monplaisir}}

\def\xxcompetences{%
\textsl{%
\textbf{Savoirs et compétences :}\\
%Les sources sont associées par un \emph{hacheur série}. La détermination des grandeurs électriques associées à ce montage permet de conclure vis à vis du cahier des charges.
%\noindent \textbf{Résoudre :} à partir des modèles retenus :
%\begin{itemize}[label=\ding{112},font=\color{ocre}] 
%\item choisir une méthode de résolution analytique, graphique, numérique;
%\item mettre en \oe{}uvre une méthode de résolution.
%\end{itemize}
%\begin{itemize}[label=\ding{112},font=\color{ocre}] 
%\item \textit{Rés -- C1.1 :} Loi entrée sortie géométrique et cinématique -- Fermeture géométrique.
%\end{itemize}
%
%\noindent \textit{Mod2 -- C4.1 :} Représentation par schéma bloc.
}}

\def\xxfigures{
\includegraphics[width=.6\linewidth]{images/fig_01}
}%figues de la page de garde


\def\xxpied{%
Cycle 99 -- Révisions\\% afin de valider leurs performances.\\
Dynamique moto -- \xxactivite%
}

\setcounter{secnumdepth}{5}
%---------------------------------------------------------------------------

\usepackage{pgfplots}
\begin{document}
\def\pathfig{images}
%\chapterimage{png/Fond_Cin}
\input{style/new_pagegarde}
\vspace{4.5cm}
\pagestyle{fancy}
\thispagestyle{plain}

\def\columnseprulecolor{\color{ocre}}
\setlength{\columnseprule}{0.4pt} 

\def\pathfig{images}

\ifprof
%\begin{multicols}{2}
\else
\begin{multicols}{2}
\fi


\subsection*{Étude de la phase d’accélération}
\setcounter{subparagraph}{0}
Une moto est modélisée par ses roues arrières et avant (notées 1 et 2) d'une part et le reste de la moto et le pilote (noté M). Les roues sont en liaison pivot avec la moto et sont en contact ponctuel avec le sol. On note $S=\{1,2,M\}$.

La phase d’étude envisagée concerne un démarrage énergique sur le premier rapport de la transmission, sur une chaussée horizontale. La figure suivante représente la moto et son pilote dans cette situation. Le couple appliqué par la transmission à la roue arrière est noté $\vect{C_{r1}}=-C_{r1}\vect{z}$


\begin{center}
\includegraphics[width=\linewidth]{images/fig_02}
\end{center}


Le coefficient de frottement au contact roue sol est supposé constant, égal à $f=0,7$. La résistance de l’air est supposée négligeable (vitesse modérée).

On suppose en première approximation que le mouvement est une translation rectiligne.

Dans ces conditions particulières, le principe fondamental de la dynamique appliqué à la moto $S$ s’écrit sous la forme simplifiée suivante :
$\torseurstat{T}{\text{ext}}{S}=\torseurcin{D}{S}{R_0}=\torseurl{m\vectg{G}{S}{R_0}}{\vect{0}}{G}$.

Les masses et inertie des roues sont faibles comparées à la masse de l’ensemble (moto -- pilote). On admet que les grandeurs dynamiques qui leur sont appliquées sont négligeables, ce qui permet de les étudier en statique.

\subparagraph{}
\textit{Faire un schéma cinématique simplifié et un graphe de structure.}
\ifprof
\begin{corrige}
\end{corrige}
\else
\fi


\subparagraph{}
\textit{Appliquer le PFD à la roue avant. En déduire la direction de l'action de contact entre la roue et le sol.}
\ifprof
\begin{corrige}

\end{corrige}
\else
\fi



\subparagraph{}
\textit{Appliquer le PFD à la roue arrière. En déduire la relation reliant le couple $C_{r1}$ et la composante tangentielle $T_1$ de l'action de contact entre la roue et le sol.}
\ifprof
\begin{corrige}
\end{corrige}
\else
\fi




\subparagraph{}
\textit{Appliquer le PFD à $S$. En déduire en fonction de $C_{r1}$ et des paramètres géométriques : 
\begin{itemize}
\item la valeur de l'accélération;
\item les composantes des actions de contact du sol sur la roue arrière en $I_1$ et sur la roue avant en $I_2$. Préciser à quelle condition le contact est sans glissement au niveau de la roue arrière. 
\end{itemize}}
\textit{Application numérique : 
$m=\SI{340}{kg}$, $a = \SI{670}{mm}$, $b = \SI{880}{mm}$, 
empattement : $L = \SI{1550}{mm}$, $C_{r1} = \SI{160}{Nm}$, $R = \SI{300}{mm}$ (Rayon de la roue) $h = \SI{700}{mm}$.}
\ifprof
\begin{corrige}
\end{corrige}
\else
\fi


\subparagraph{}
\textit{Quel compromis peut-on faire entre le couple $C_{r1}$, la position du centre de gravité suivant $\vect{x}$ et le coefficient de frottement afin de ne pas déraper et d’éviter un wheeling.}
\ifprof
\begin{corrige}
\end{corrige}
\else
\fi

\newpage 

\subsection*{Étude de la moto en courbe}
La figure suivante représente la moto et son pilote en vue de derrière lors d’un passage en courbe.

\begin{center}
\includegraphics[width=\linewidth]{images/fig_03}
\end{center}



On suppose que la moto roule à vitesse uniforme, dans une courbe à rayon constant $R_c$, et que la limite du glissement est atteinte au niveau du contact des roues avec le sol.

Dans ces conditions, le principe fondamental de dynamique s’écrit de la façon suivante : 

$\torseurstat{T}{\text{ext}}{S}=\torseurcin{D}{S}{R_0}=\torseurl{m\vectg{G}{S}{R_0}}{\vect{0}}{G}$.

Le coefficient de frottement au contact roue sol est supposé constant, égal à $f = 0,7$.
La résistance de l’air est supposée négligeable (vitesse modérée).
On considère une étude dynamique dans le plan  $\left(G,\vect{y},\vect{z} \right)$.



\subparagraph{}
\textit{Calculer l’accélération du centre de gravité $G$ de l’ensemble $M$ en fonction de la vitesse $V$ de la moto, supposée constante et du rayon $R_c$ du virage.}
\ifprof
\begin{corrige}
\end{corrige}
\else
\fi



\subparagraph{}
\textit{Appliquer le PFD à $E$. En déduire : 
\begin{itemize}
\item l'inclinaison de la moto;
\item la vitesse $V$ de passage dans le virage en fonction de son rayon $R_c$;
\item l'action globale du sol sur les roues en $I_1$.
%\item valeurs de l'effort au contact roue arrière -- sol. 
\end{itemize}}
%
%\subsection*{Bilan}
%
%\subparagraph{}
%\textit{Préciser celle des deux situations précédentes qui induit l’effort le plus élevé au contact roue arrière/sol.}
%\ifprof
%\begin{corrige}
%\end{corrige}
%\else
%\fi


\ifprof
%\end{multicols}
\else
\end{multicols}
\fi

\end{document}	