\documentclass[10pt,fleqn]{article} % Default font size and left-justified equations
\usepackage[%
    pdftitle={Modélisation dynamique},
    pdfauthor={Xavier Pessoles}]{hyperref}

    
\input{style/new_style}
\input{style/macros_SII}
\usepackage{multicol}
\usepackage{siunitx}
%\usepackage{picins}
\fichetrue
%\fichefalse

\proftrue
\proffalse

\tdtrue
%\tdfalse

\courstrue
\coursfalse


\def\classe{\textsf{PSI$\star$ -- MP}}
\def\xxnumpartie{Cycle 99}
\def\xxpartie{Révisions}

\def\xxnumchapitre{}%Chapitre 1 \vspace{.2cm}}
\def\xxchapitre{}%\hspace{.12cm} Approche énergétique}

\def\discipline{Sciences \\Industrielles de \\ l'Ingénieur}
\def\xxtete{Sciences Industrielles de l'Ingénieur}




\def\xxtitreexo{Robovolc : un robot pour l'exploiration volcanique}
\def\xxsourceexo{\hspace{.2cm} \textsl{Équipe PT La Martinière Monplaisir -- X ENS PSI}
}


\def\xxposongletx{2}
\def\xxposonglettext{1.45}
\def\xxposonglety{20}
%\def\xxonglet{Part. 1 -- Ch. 3}
\def\xxonglet{\textsf{Cycle 99}}

\def\xxactivite{Application}
\def\xxauteur{\textsl{Équipe PT La Martinière Monplaisir}}

\def\xxcompetences{%
\textsl{%
\textbf{Savoirs et compétences :}\\
%Les sources sont associées par un \emph{hacheur série}. La détermination des grandeurs électriques associées à ce montage permet de conclure vis à vis du cahier des charges.
%\noindent \textbf{Résoudre :} à partir des modèles retenus :
%\begin{itemize}[label=\ding{112},font=\color{ocre}] 
%\item choisir une méthode de résolution analytique, graphique, numérique;
%\item mettre en \oe{}uvre une méthode de résolution.
%\end{itemize}
%\begin{itemize}[label=\ding{112},font=\color{ocre}] 
%\item \textit{Rés -- C1.1 :} Loi entrée sortie géométrique et cinématique -- Fermeture géométrique.
%\end{itemize}
%
%\noindent \textit{Mod2 -- C4.1 :} Représentation par schéma bloc.
}}

\def\xxfigures{
\includegraphics[width=.4\linewidth]{images/fig_00}
}%figues de la page de garde


\def\xxpied{%
Cycle 99 -- Révisions\\% afin de valider leurs performances.\\
Robovolc : un robot pour l'exploiration volcanique -- \xxactivite%
}

\setcounter{secnumdepth}{5}
%---------------------------------------------------------------------------

\usepackage{pgfplots}
\begin{document}
\def\pathfig{images}
%\chapterimage{png/Fond_Cin}
\input{style/new_pagegarde}
\vspace{4.5cm}
\pagestyle{fancy}
\thispagestyle{plain}

\def\columnseprulecolor{\color{ocre}}
\setlength{\columnseprule}{0.4pt} 

\def\pathfig{images}

\ifprof
%\begin{multicols}{2}
\else
\begin{multicols}{2}
\fi


\subsection*{Présentation du système}

\setcounter{subparagraph}{0}

\subsubsection*{Contexte}

Les observations scientifiques réalisées pendant les phases éruptives sont aujourd'hui fondamentales pour l'étude des volcans. En effet, les prélèvements des gaz magmatiques et des échantillons rocheux rejetés lors de ces phases constituent des indicateurs fiables de l'activité interne des volcans ; ils sont donc une riche source d'informations pour les volcanologues. Cependant, les phases éruptives sont aussi des phases actives très dangereuses et il est primordial de limiter les risques humains lors d'observations et de prélèvements à proximité des cratères en éruption.

Avec ce constat, allié aux avancées technologiques dans le domaine de la robotique, la Communauté Européenne a financé le projet ROBOVOLC dont le but était la réalisation d'un robot mobile pour l'exploration volcanique. Ce robot devait être capable de :
\begin{itemize}
\item s'approcher d'un cratère actif ;
\item collecter des échantillons rocheux issus de rejets éruptifs;
\item collecter des échantillons gazeux;
\item collecter d'autres données physiques et chimiques.
\end{itemize}

Le système ROBOVOLC est représenté sur la figure suivante. Il se divise en plusieurs sous-systèmes (liés à la navigation, au prélèvement et à la communication) qui sont détaillés dans les diagrammes SysML fournis dans l'Annexe 1.

\begin{center}
\includegraphics[width=\linewidth]{images/fig_02}
\textit{Représentation du système ROBOVOLC}
\end{center}


La partie mécanique de ROBOVOLC est constituée de deux parties : (i) la plateforme (châssis, roues) servant à la locomotion ; (ii) l'équipement d'analyse (bras manipulateur, pince, sondes) pour le prélèvement et la mesure.

Une contrainte particulière dans la conception du système ROBOVOLC est qu'il est soumis à des conditions extérieures particulièrement difficiles : terrain volcanique non structuré avec obstacles et fortes pentes, températures très élevées près des zones éruptives (les gaz atteignent 600°C) mais basses ailleurs à cause de l'altitude, présence de poussières de cendre très fines, ambiance corrosive due aux composants acides, etc.


\begin{center}
\includegraphics[width=\linewidth]{images/fig_03}
\textit{Illustration du pilotage à distance du système ROBOVOLC}
\end{center}

Un cahier des charges partiel est donné ci-dessous :
\begin{center}
\begin{tabular}{|p{5cm}|p{2cm}|}
\hline
Critère & Valeur \\ \hline
Temps de trajet pour une mission de 24 heures & 1,5 \\ \hline
Vitesse de déplacement atteignable	& \SI{0,5}{m/s} \\ \hline
Dimensions du système (longueur/largeur/hauteur)& 1900 mm x 1200 mm x 800 mm \\ \hline
Masse maximale des composants modulaires & \SI{200}{kg}\\ \hline
Charge utile maximale (instruments, etc.)	& \SI{30}{kg} \\ \hline
Pente maximale du sol	& 40\degres \\ \hline
Hauteur maximale d'un obstacle 	& \SI{400}{mm}\\ \hline
\end{tabular}
\end{center}

\subsection*{Étude de la stabilité du Robovolc}
Les basculements frontal et latéral du Robovolc représentent le principal risque auquel est confronté le véhicule. L’objectif de cette partie est de définir les conditions de stabilité du chariot élévateur dans les phases de freinage et lors des virages afin de définir les conditions optimales de déplacement du Robovolc dans chacune de ces phases.

\subsubsection*{Étude de la position du centre de gravité}

\begin{center}
\includegraphics[width=\linewidth]{images/fig_04}
%\textit{Illustration du pilotage à distance du système ROBOVOLC}
\end{center}


\begin{itemize}
\item L’articulation entre le bras 3 et le châssis est modélisée par une liaison pivot glissant d’axe $\axe{O_1}{Z_1}$.
\item L’articulation entre le bras 3 et l'avant-bras 4 est modélisée par une liaison pivot glissant d’axe $\axe{O_3}{Z_1}$.
\end{itemize}

Données géométriques : $\vect{OG_1}=a_1\vect{X_1}+h\vect{Z_1}$, $\vect{OG_2}=a_2\vect{X_1}+H\vect{Z_1}$, $\vect{OO_1}=L_1\vect{X_1}+\lambda\vect{Z_1}$, $\vect{O_1G_3}=\dfrac{L_3}{2}\vect{X_3}$, $\vect{O_1O_3}=L_3\vect{X_3}$, $\vect{O_3G_4}=\dfrac{L_3}{2}\vect{X_4}$.

\begin{itemize}
\item $G_1$ est le centre d’inertie du châssis et des 6 roues : masse $m_1$.
\item $G_2$ est le centre d’inertie du corps du robot : masse $m_2$.
\item $G_3$ est le centre d’inertie du bras 3 du robot : masse $m_3$.
\item $G_4$ est le centre d’inertie de l’avant-bras 4 et de la charge du robot : masse $m_4$.
\end{itemize}



\subparagraph{}
\textit{Déterminer les coordonnées $X_G$, $Y_G$, et $Z_G$ du centre de gravité $G$ de l’ensemble du Robovolc dans le repère $R_1$.}
\ifprof
\begin{corrige}
\end{corrige}
\else
\fi


Pour la suite du problème. Le point $G$ de l’ensemble du Robovolc (charge comprise) est définit par : $\vect{OG}=X_G\vect{X_1}+Y_G\vect{Y_1}+Z_G\vect{Z_1}$ avec $M$ la masse totale du Robovolc (environ \SI{250}{kg}).


\subsubsection*{Étude du basculement frontal (problème 2D)}
Le freinage est une phase délicate de stabilité pendant laquelle le Robovolc peut basculer. Dans cette phase, le Robovolc freine en urgence et décélère tel que $\vect{\Gamma(G/0)}=-a_{cc}\vect{X_0}$ sur un sol horizontal.

\begin{hypo}
Les contacts en $O$ et $I_2$ sont modélisés par des liaisons ponctuelles et il n’y pas de contact en $I_1$. La gravité $g$ est portée par $-\vect{Z_0}$.

\end{hypo}


\begin{center}
\includegraphics[width=\linewidth]{images/fig_05}
%\textit{Illustration du pilotage à distance du système ROBOVOLC}
\end{center}

\subparagraph{}
\textit{Déterminer la valeur maximale de $a_{cc}$ pour que le Robovolc ne bascule pas vers l’avant.}
\ifprof
\begin{corrige}
\end{corrige}
\else
\fi

\subsubsection*{Étude du basculement latéral}
Le centre de gravité du Robovolc complet $\vect{O_0 G} = -\rho_G\vect{X_1}+Z_G\vect{Z_1}$ avec $M$ la masse totale du Robovolc (environ \SI{250}{kg}). 
Le Robovolc prend un virage à gauche de rayon $R$ à une vitesse $V$ constante c'est-à-dire que la trajectoire du point $O_1$ est un cercle de centre $O_0$ et rayon $R$ et sa vitesse $\vectv{O_1}{1}{0}=V\vect{X_1}$.

\begin{hypo}
Le problème est abordé comme un problème plan sur la vue $\left(O_1;\vect{Y_1};\vect{Z_1}\right)$. Les actions du sol sur les roues gauches et sur les roues droites sont modélisées par deux liaisons ponctuelles en $I_G$ et $I_D$ visibles sur le plan $\left(O_1;\vect{Y_1};\vect{Z_1}\right)$.

Pour l’inertie la matrice du Robovolc complet sera considérée comme diagonale malgré la dissymétrie du bras.

\end{hypo}

\begin{center}
\includegraphics[width=\linewidth]{images/fig_06}
%\textit{Illustration du pilotage à distance du système ROBOVOLC}
\end{center}


\subparagraph{}
\textit{Quel théorème doit-on utiliser afin d’obtenir directement l’équation permettant de déterminer l’expression de $V$ qui provoque le basculement latéral ?
En déduire l’expression de $V$ qui provoque le basculement latéral du chariot.}
\ifprof
\begin{corrige}
\end{corrige}
\else
\fi


\subsection*{Étude des capacités d'accélération du Robovolc dans une forte pente}

Le robot évolue sur de fortes pentes et s’arrête et redémarre fréquemment pour prélever des échantillons. Cette partie permet de valider la capacité du Robovolc à accélérer sur la pente d’un volcan.


\subsubsection*{Étude des performances en montée}
Dans cette sous-partie on analyse le Robovolc lors du franchissement d'une montée.
L'objectif est de déterminer les performances du Robovolc en montée en fonction des caractéristiques des moteurs.

On supposera que toutes les roues sont motrices en contact avec le sol et qu'il n'y a pas de glissement des roues sur le sol.

\begin{center}
\includegraphics[width=\linewidth]{images/fig_07}
%\textit{Illustration du pilotage à distance du système ROBOVOLC}
\end{center}


La gravité $g$ est portée par $-\vect{z_0}$.


On note :
\begin{itemize}
	\item $V$ : la vitesse de déplacement du Robovolc;
	\item $M_C$ : la masse du châssis et du bras (avec sa charge) sans les roues (\SI{190}{kg});
	\item $\omega_r$ : la vitesse de rotation des roues par rapport au châssis;
	\item $C_r$ : le couple des moteurs sur les roues (\SI{40}{N.m});
	\item $M_R$ la masse d'une roue (\SI{10}{kg});
	\item $J_R$  : l'inertie d'une roue par rapport à son axe de rotation $\axe{C_i}{Y_1}$ :  $J_R=\SI{0,11}{kg.m^2}$;
	\item $R_{\text{roue}}$ : le rayon de la roue $R_{\text{roue}}=\SI{150}{mm}$.
	\item on note $C_0$, $C_1$ et $C_2$ les centres des liaisons pivots entre les roues et le châssis.
\end{itemize}



\subparagraph{}
\textit{Sur le document réponse, dessiner le vecteur représentant le poids ainsi que les actions du sol sur les roues. (Seuls l'orientation et le sens sont demandés, la norme du vecteur n'est pas demandée). }
\ifprof
\begin{corrige}
\end{corrige}
\else
\fi


\subparagraph{}
\textit{Écrire la condition de roulement sans glissement en $O$ et par composition de vitesse en déduire une relation entre $V$ et $\Omega$.}
\ifprof
\begin{corrige}
\end{corrige}
\else
\fi



\subparagraph{}
\textit{En appliquant le Théorème de l'énergie cinétique à l'ensemble du Robovolc, déterminer l'équation du mouvement du Robovolc. Écrire l'énergie cinétique sous la forme:  $T\left(\text{Robovolc}/0\right)=\dfrac{1}{2} M_{eq}V^2$.}
\ifprof
\begin{corrige}
\end{corrige}
\else
\fi

\subsection*{Étude du comportement sur terrain accidenté}

\begin{obj}
L'objectif de cette partie est de valider les performances d'agilité et de franchissement d'obstacle du système sur des terrains non structurés avec difficultés topologiques (pentes, obstacles).
\end{obj}

Dans toute cette partie, les effets dynamiques sont négligés et on se place en statique. 

La mobilité sur terrain accidenté est obtenue par l'utilisation d'un châssis articulé, appelé ici plateforme, permettant à ROBOVOLC de s'adapter à toute surface non plane.


Les deux articulations EC-AAV et EC-AAR, situées à une distance longitudinale $\pm b$ de l'essieu EC, autorisent une rotation selon les directions $\vect{x}$ et $\vect{y}$ mais ne permettent pas de rotation autour de l’axe $\axe{C}{z}$ et $\axe{D}{z}$, elles sont modélisées par des liaisons rotule à doigt de centres respectifs $B$ et $C$.

Les deux articulations EAV-AAV et EAR-AAR autorisent une rotation selon la direction $\vect{x}$ seulement; elles sont modélisées par des liaisons pivot d'axe $\axe{O}{x}$.

D'autre part, les six liaisons essieu-roue sont modélisées par des liaisons pivot d'axe $\axe{A}{y}$ (roues avant), $\axe{O}{y}$ (roues centrales) ou $\axe{D}{y}$ (roues arrière). De plus, le contact de chaque roue $i$ avec le sol est modélisé en première approche par une liaison ponctuelle de normale $\axe{P_i}{z}$ sans frottement.


\subparagraph{}
\textit{Déterminer le nombre de mobilités du modèle de la plateforme.}
\ifprof
\begin{corrige}
\end{corrige}
\else
\fi

\subparagraph{}
\textit{Montrer que ce modèle est isostatique. Conclure quant à la capacité du châssis à maintenir les roues au contact du sol en toute circonstance.}
\ifprof
\begin{corrige}
\end{corrige}
\else
\fi


\ifprof
%\end{multicols}
\else
\end{multicols}
\fi


\begin{center}
\includegraphics[width=\linewidth]{images/fig_11}
%\textit{Illustration du pilotage à distance du système ROBOVOLC}
\end{center}

\begin{center}
\includegraphics[width=\linewidth]{images/fig_12}
%\textit{Illustration du pilotage à distance du système ROBOVOLC}
\end{center}


\end{document}	


\subparagraph{}
\textit{}
\ifprof
\begin{corrige}
\end{corrige}
\else
\fi
